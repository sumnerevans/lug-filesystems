\documentclass{lug}

\usepackage{fontawesome}
\usepackage{etoolbox}
\usepackage{etoolbox}
\usepackage{textcomp}
\usepackage[nodisplayskipstretch]{setspace}
\usepackage{xspace}

\AtBeginEnvironment{minted}{\singlespacing\fontsize{10}{10}\selectfont}

\makeatletter
\patchcmd{\beamer@sectionintoc}{\vskip1.5em}{\vskip0.5em}{}{}
\makeatother

\title{Filesystems}
\author{Sumner Evans and Sam Sartor}
\institute{Mines Linux Users Group}

\begin{document}

\section{Introduction}

\begin{frame}{What are Filesystems?}
    \begin{itemize}
        \item Filesystems manage the storage and retrieval of files from storage
            media.
        \item Filesystems are an abstraction layer between storage media (SSDs,
            HDDs, disk drives, even tape drives).
        \item Filesystems exist on \textit{partitions}, physically contiguous
            segments of the disk.
    \end{itemize}
\end{frame}

\begin{frame}{Filesystems are Responsible for\ldots}
    \begin{itemize}
        \item \textbf{Space management:} filesystems allocate and manage space
            in discrete chunks. Filesystems must keep track of what data is
            stored at each chunk.
        \item \textbf{Filenames:} identify a storage location in the file
            system. Can be case sensitive (ext4) or case insensitive (HFS,
            NTFS).
        \item \textbf{Directories (folders):} group files into separate
            collections. Modern filesystems allow arbitrary nesting of
            directories.
        \item \textbf{Metadata:} filesystems store book-keeping information
            about their contents (e.g. file sizes, last accessed date, owner and
            permissions, etc.).
        \item \textbf{Access Control:} prevent unauthorized access to files on
            disk.
        \item \textbf{Data Integrity:} filesystems must be resilient to failure,
            some are better at this than others.
    \end{itemize}
\end{frame}

\section{History of Filesystems}

\section{Current Filesystems}

\begin{frame}[standout]
    \Huge
    Questions?
\end{frame}

\begin{frame}{References}
    \begin{itemize}
        \item \url{https://en.wikipedia.org/wiki/File_system}
        \item \url{http://www.tldp.org/LDP/sag/html/filesystems.html}
    \end{itemize}
\end{frame}

\end{document}
